\section{Surface reconstruction}
Surface reconstruction from a large set of unstructured points obtained through laser scanning techniques or stereopsis is not a trivial task but nonetheless it is useful in many applications. For example a surface reconstruction could aid a robots end effector grasping some unknown object \cite{Wang2005}. Many methods for surface reconstruction exist in different domains of science, some algorithms are based on neural network approaches  \cite{Wu2007}, others are based on sculpting or region growing algorithms (e.g \cite{Bernardini1999} and \cite{Amenta2001}) used in computational geometry, some utilise an implicit method framework to represent incoming data as a surface \cite{Kazhdan2006} and \cite{Dong2011}.\\
\\
In 1999 F. Bernardini \textit{et al}. proposed in a fairly simple algorithm (Ball Pivoting Algorithm (BPA)) for reconstruction of surfaces from point clouds sampled over smooth objects \cite{Bernardini1999}. BPA is pivoting a sphere of a certain diameter around an edge of a seed triangle. Pivoting the sphere around all the edges is connecting three points to form a new triangle and so on. BPA is a part of the region growing algorithms as this algorithm uses a seed triangle builds the surface around this and outward. The algorithm is fairly easy to work with as it has one parameter which decides the radius of the sphere and that is it. N. Amenta \textit{et al}. proposed in 2001 \cite{Amenta2001} the power crust algorithm, which essentially is a three dimensional Voronoi approach \cite{Ledoux2007}. The power crust algorithm is well defined and proven which makes it one of the most known algorithm regarding surface reconstruction. This algorithm is a sculpting algorithm in computational geometry. Common for those algorithms mentioned is that these are explicit methods which requires neighboring information which leads to high computational time consumption. Therefore methods mentioned above are not well suited for reconstruction of large data sets.\\
\\
In 2006, M. Kazhdan \textit{et al}. proposed in \cite{Kazhdan2006} an algorithm which resides in the implicit method framework, this algorithm is called Poisson Surface Reconstruction (PSR). PSR is a fitting scheme that allow solving for the indicator function of the surface. It is shown in \cite{Kazhdan2006} this approach of fitting a surface resembles the Fast Fourier Transform (FFT). In \cite{Kazhdan2006} it is shown that the FFT approach requires five times as much space but is approximately double as fast while creating approximately the same number of triangles. The real advantage of the Poisson approach is that it is scalable and therefore does not rely on a uniform distribution of points and thus a higher degree of details can be reconstructed in areas where the point density is higher.\\
J. Manson \textit{et al}. did in 2008 propose a wavelet approach of the problem of reconstructing surfaces of large sets of unstructured points \cite{Manson2008}. The method is robust to noise in data because it is relying on implicit methods as PSR, and is therefore well-suited for reconstruction of real data which is overlayed with some random noise. The wavelet approach an advantages over PSR, the wavelet approach is able to reconstruct non-closed surfaces in opposition to PSR \cite{Kazhdan2006}. Using the Haar wavelet synthesise an acceptable surface if the points are well-aligned, else more smooth wavelets can be used such as Daubechies-2. Using octrees and small-support wavelets makes the synthesisation very fast, smaller support equals less computational time and space. Using Daubechies-2 compared to Haar increases computational time by a scale of four, and the space consumption is upscale by approximately four as well. As the methods for reconstruction mentioned first does not guarantee watertight mesh and does not work very well with large point sets, mean those methods are not of interest for further studies. The PSR and the wavelet approach both does estimate an indicator function of the surface, but each handles this indicator function differently. As these method is based on implicit methods they are more robust to noise in the input and approximates the surfaces better compared to the methods mentioned first.

\subsection{Overview of reconstruction scheme}
The wavelet method for surface reconstruction chosen for this custom implementation is highly inspired by the description in \cite{Manson2008}. One main reason for using wavelets for the analysis of the surface compared to more conventional methods like FFT is that wavelets support localisation in spatial and frequency(scale) domains. This mean that analysis is made locally instead of globally, which leads to the possibility of reconstructing finer details in input data, that would otherwise be hidden by the global analysis. PSR is approximating an indicator function\marginnote{ref: indicator function} of the surface by solving a Poisson equation\marginnote{ref: Poisson eq.}. In the wavelet approach an indicator function is also estimated, but the way it is done is different. The wavelet approach approximates the wavelet coefficients of the indicator function by making a multi resolution analysis (MRA) of the input data\marginnote{ref: MRA, coeffs}. After the coefficients have been calculated ...

\subsection{Indicator function}
The method mentioned in \cite{Manson2008} defines the indicator function of the surface $\partial M$ of an object $M$ to be as follows

\begin{equation}
	\chi_M(\textbf{x}) =  
	\label{eq:indicator_fnc}
\end{equation}



\begin{equation}
	c^e_{j,\textbf{k}} \approx 2^{3j \over 2} \sum_{p_i \in \partial M \cap \textit{supp}(\psi^e_{j,\textbf{k}})} \vec{F}^e_{j,\textbf{k}} (p_i) \cdot \vec{n}_i d \sigma_i 
	\label{eq:wavelet_coeffs}
\end{equation}

\subsection{Mesh generation}

\subsection{Implementation}
The surface reconstruction is implemented in ROS node making it compatible with the rest of the system. The assembled point cloud is passed from the filter node to the reconstruction node. PCL is used for estimating normals of the points in the assembled point cloud. 

\subsection{Results}