\section{Surface reconstruction}
Surface reconstruction from a large set of unstructured points obtained through laser scanning techniques or stereopsis is not a trivial task but nonetheless it is useful in many applications. For example a surface reconstruction could aid a robots end effector grasping some unknown object \cite{Wang2005}. Many methods for surface reconstruction exist in different domains of science, some algorithms are based on neural network approaches  \cite{Wu2007}, others are based on sculpting or region growing algorithms (e.g \cite{Bernardini1999} and \cite{Amenta2001}) used in computational geometry, some utilise an implicit method framework to represent incoming data as a surface \cite{Kazhdan2006} and \cite{Dong2011}.\\
In 1999 F. Bernardini et al. proposed in a fairly simple algorithm (Ball Pivoting Algorithm (BPA)) for reconstruction of surfaces from point clouds sampled over smooth objects \cite{Bernardini1999}. BPA is pivoting a sphere of a certain diameter around an edge of a seed triangle. Pivoting the sphere around all the edges is connecting three points to form a new triangle and so on. BPA is a part of the region growing algorithms as this algorithm uses a seed triangle builds the surface around this and outward. The algorithm is fairly easy to work with as it has one parameter which decides the radius of the sphere and that is it. N. Amenta et al. proposed in 2001 \cite{Amenta2001} the power crust algorithm, which essentially is a three dimensional Voronoi approach \cite{Ledoux2007}. The power crust algorithm is well defined and proven which makes it one of the most known algorithm regarding surface reconstruction. This algorithm is a sculpting algorithm in computational geometry. Common for those algorithms mentioned is that these are explicit methods which requires neighboring information which leads to high computational time consumption. Therefore methods mentioned above are not well suited for reconstruction of large data sets.\\
In 2006 M. Kazhdan et al. proposed in \cite{Kazhdan2006} an algorithm which resides in the implicit method framework, this algorithm is called Poisson Surface Reconstruction (PSR). PSR is a fitting scheme that allow solving for the indicator function of the surface. It is shown in \cite{Kazhdan2006} this approach of fitting a surface resembles the Fast Fourier Transform (FFT). In \cite{Kazhdan2006} it is shown that the FFT approach requires five times as much space but is approximately double as fast while creating approximately the same number of triangles. The real advantage of the Poisson approach is that it is scalable and therefore does not rely on a uniform distribution of points and thus a higher degree of details can be reconstructed in areas where the point density is higher.\\
J. Manson et al. did in 2008 propose a wavelet approach of the problem of reconstructing surfaces of large sets of unstructured points \cite{Manson2008}. The method is robust to noise in data because it is relying on implicit methods as PSR, and is therefore well-suited for reconstruction of real data which is overlayed with some random noise. The wavelet approach an advantages over PSR, the wavelet approach is able to reconstruct non-closed surfaces in opposition to PSR \cite{Kazhdan2006}. Using the Haar wavelet synthesise an acceptable surface if the points are well-aligned, else more smooth wavelets can be used such as Daubechies-2. Using octrees and small-support wavelets makes the synthesisation very fast, smaller support equals less computational time and space. Using Daubechies-2 compared to Haar increases computational time by a scale of four, and the space consumption is upscale by approximately four as well. As the methods for reconstruction mentioned first does not guarantee watertight mesh and does not work very well with large point sets, mean those methods are not of interest for further studies. The PSR and the wavelet approach both does estimate an indicator function of the surface, but each handles this indicator function differently. As these method is based on implicit methods they are more robust to noise in the input and approximates the surfaces better compared to the methods mentioned first.