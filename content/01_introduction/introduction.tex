\chapter{Introduction}
Electronic representations of three-dimensional objects are of great importance to many applications. Often the 3D models are hand-build which can be a tedious and time consuming task thus making automatic modelling desirable. In robotics, models of the environment are used as a priori knowledge for collision checking. The problem with hand-build models is that they limit the application of robots to known environments. \\

One method for recording 3D models is to mount a stereo camera on the end effector of a robot arm (eye-in-hand) and let the robot move the camera to the views needed to generate the model. This way the robot is able to assess unknown objects and update the work cell model accordingly. The precision of the model is not only dependent on the quality of robot arm and camera, but also on the calibration of the system. The camera can be calibrated using a marker plate in different poses thus calculating intrinsic parameters and camera disparity \cite{Zhang2000}. The robot model can be calibrated in a process called hand-eye calibration solving for the unknown spatial relationships in the kinematic chain.\\

Another important part of 3D modelling is the reconstruction problem, reasoning about either surface, volume or both. Depending on the application there are very different demands for the final model. If it is used for collision detection a coarse model is more efficient, while a model for grasping needs to be sufficiently detailed to infer grasping points. Modelling is a multi-step process where the gathered data is first transformed, filtered and combined to form a point cloud representing the object. Next an indicator function is approximated to best fit the normals of the inferred solid. Finally a mesh is generated from the indicator function. Using wavelets to approximate the indicator function provides a localised, multi-resolution representation and thus addresses the trade-off between faster coarse models and slower fine models.\\

In this work an eye-in-hand system for wavelet based surface reconstruction is developed and the effect of hand-eye calibration is investigated. It is hypothesised that a calibrated system can generate significantly better models than the same system without calibration. The calibration is evaluated in a manipulative study, where a known object is modelled before and after calibrating the system. The reconstructed models are visually as well as quantitatively compared to the true 3D model of the object. \\

The novelties in this report are hand-eye calibration based on point cloud data, an implementation of the wavelet based surface reconstruction proposed in \cite{Someone wielding the wavelet hammer...} and implementing the entire system in Robot Operating System(ROS).\\

In chapter 2 the system is analysed on component level to formulate a requirements specification forming the basis for chapter 3 describing decision making, chapter 4 describing robotics and chapter 5 describing computer vision. In chapter 6 the principles and implementation of wavelet based 3D modelling in ROS are described and chapter 7 is about the point cloud based calibration system. Finally in chapter 8 the evaluation method and findings are presented and discussed, leading to a conclusion in chapter 9.