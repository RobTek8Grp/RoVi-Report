\chapter{Introduction}

Models are necessary in most applications of robotics. Often the models are build in advance and used by the robot as a priori knowledge, but this approach limits the application of robots in dynamic environments. Furthermore modelling complex environments can be a tedious and time consuming task, making automatic modelling desirable. 

One way of recording such 3D models is by mounting a stereo camera on the end effector of a robot arm and let the robot move the camera to the views needed to generate the model. The precision of such systems is dependent on robust calibration with respect to both camera and kinematics.

One way of recording such 3D models is by mounting a stereo camera on the end effector of a robot arm and let the robot move the camera to the views needed to generate the model. The precision of the model is dependent on robust calibration with respect to both camera and kinematics. The camera can be calibrated using a marker plate in different poses thus calculating intrinsic parameters and camera disparity \cite{Zhang2000}. The robot pose can be calibrated in a process called hand-eye calibration solving for the unknown spatial relationships in the kinematic chain. 

It is hypothesised that a hand-eye calibrated system can generate significantly more precise models than the same system without calibration.

Evaluation of the 3D model is based on a known object, where the combined point cloud can be compared to the 3D model of the object. This introduces a pose estimation problem, since the model must be aligned to the point cloud. 

Calibration of robot systems has received considerable attention and continues to be an active field of research. A solution for the unknown transforms from camera to end effector and from maker to robot base can be obtained relatively easy by solving a homogeneous transform \cite{Shiu1989}. Several algorithms for more or less autonomous calibration has been proposed \cite{Tsai1988, Tsai1989} often simultaneously calibrating camera and hand-eye \cite{Malm2003, Zhao2008, Jordt2009}.

This work is a manipulative study investigating the effect of hand-eye calibration measured on the quality of the produced point cloud. The novelty of the study is the practical implementations of hand-eye calibration and model evaluation as well as calibration routines for the robot kinematics.


%\input{content/01_introduction/state_of_the_art.tex}