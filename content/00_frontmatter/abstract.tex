\chapter*{Abstract}\addcontentsline{toc}{chapter}{Abstract}
Digital representations of three-dimensional objects are of great importance to many applications. Often the 3D models are hand-build which can be a tedious and time consuming task thus making automatic modelling desirable. In robotics, models of the environment are used as a priori knowledge for collision checking. The problem with hand-build models is that they limit the application of robots to known environments. \\

One method for recording 3D models is to mount a stereo camera on the end effector of a robot arm (eye-in-hand) and let the robot move the camera to the views needed to generate the model. This way the robot is able to assess unknown objects and update the work cell model accordingly. In this work we propose a calibration method based on point clouds and we present a fully functional system based on ROS and build from scratch. Development of the system was aimed at investigating the hypothesis that a locally calibrated system could generate significantly better models than the same system without calibration and that basing the entire system on the ROS framework would reduce development time. \\

Evaluation of the system was based on a comparison between the hand-eye calibration module implemented in ROS and the proposed application specific calibration. Although several experiments were made, they mostly served to get acquainted with the system and the results are insufficient to draw any final conclusions, but, some partial conclusions are stated. The calibration method worked, but performed poorly and can thus not be recommended at this point. The implementation of the system in ROS based on the environment interaction model worked well. \\