\chapter*{Preface}\addcontentsline{toc}{chapter}{Preface}
This project was conducted as part of the RoVi2 course at University of Southern Denmark (SDU) during the spring semester 2014. Since neither of us had practical experience from using a robot arm, we decided to approach the task in two parts. First we wanted to set up a fully functional system using the components offered in libraries, frameworks and operating system. The aim was to get to know the system and to have chance of pinpointing the actual challenges in the system. Next we wanted to make an application to generate 3D models based on an eye-in-hand set up.\\

As the project progressed it became clear that calibration would be a key issue and therefore we changed direction from expecting to do next best view planning and wavelet based reconstruction, to focusing on getting the basic system to perform well. We were determined to thoroughly investigate the implications of a calibrated system and to document its performance while still doing the wavelet based reconstruction, aiming to implement an algorithm described in a paper. \\

It turned out however, that we were not able to get the wavelet based algorithm to work within the given time and furthermore source code from the authors website compiled, but did not work either. We went to two different experts in wavelets and as they also failed to help, we had to abandon the idea, leaving us with only the calibration part.

Focusing only on calibration also reduced the amount of experiments that we planned to perform, as we did not want to document the performance of algorithms we did not make ourselves and likewise we did not want to evaluate things we do not use as for example the applicability of calibration in gripper applications. \\

We would have liked the report to be shorter, describing only the parts we developed ourselves and how the system is put together otherwise focusing on experiments like in a real life situation, but we had to realise that, regardless of its realism, it is still a school project.\\

Despite all the challenges we still believe that the report turned out to be reasonably good.\\
 \\
 Frederik, Rudi, Kent and Leon