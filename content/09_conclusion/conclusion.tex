\chapter{Discussion}
Although the results were insufficient to draw any final conclusions, some of the many observations and reflections are worth mentioning. Many factors and decisions has influenced the project both good and bad and therefore also deserves discussion to draw experiences.\\

There has been many advantages from implementing the entire system in ROS leading to significantly reduced development time. The rosbag tool made it possible to work offline and thus enabled parallel development of components and effectiveness despite limited access to the robot. Being able to facilitate the rviz visualisation tool provided a safe and intuitive control, since the pinhole of the camera could be placed at will. ROS also delivered increased flexibility because of close integration with many state of the art libraries and frameworks as well as support for both python and C++ programming languages. The architecture of the designed system combined with the inter process communication model provided in ROS made the interfaces clear and further enhanced effective work. Finally the vast communities of ROS, OpenCV, PCL, OMPL etc. has been invaluable.\\

There have been many challenges in the project and many of them influenced its outcome. First there were great challenges with the reach of the robot compared to the range of the sensors. Even when the object was hanged from a point above the robot, the maximum radius of the sphere was 45cm with is close to the 40cm minimal range of the carmine sensor and leading to a very large disparity of 130 on the bumblebee camera. This challenged the rectification process, increased noise on the point cloud and thus also challenged the stitching process. Hanging the object also introduced the problem of fixing the object because the work cell fence is not fixed and thus the slightest touch would make the object swing again challenging the system.\\

During experimentation it became very apparent that some implicit assumptions had been made and those made even simple tasks very difficult. As mentioned above, it was implicitly assumed that everything would be static during capture, which is never a problem when working on the Canterbury images, but is a great problem in reality. The disparity in the Canterbury set is usually around 16 and this was implicitly assumed to be the case however it turned out to be almost twenty times as large. The list is much longer but a point has been made, that even though having focused on requirements and assumptions all along, important issues were hidden from the beginning due to inexperience.\\

The vision features turned out to be a much greater challenge than expected. Even with good lighting, unique features on the object and a calibrated system, the quality of the point clouds were still too poor to do 3D reconstruction on, which underlines another implicit assumption, that reconstruction is at all possible in this set up. Although much better with the active Carmine sensor, the point clouds were still somewhat poor, and by no means matched the quality of point clouds that are presented in papers, textbooks and programming examples. The importance of features in the point cloud and the importance of sufficient overlap also became evident as did the profound differences between active and passive sensing.\\

The proposed calibration method performed as a proof of concept, although exhibiting the effects of many of the implicit assumptions. The performance in the present experiments were poor, but given the right application and further advances in pose estimation of noisy point clouds, it might hold potential. The system in general, but calibration in particular performed too poor to be evaluated in the manner planned based on comparison between point cloud and 3D model.\\

\chapter{Conclusion}
Even if a clear conclusion cannot be made at this point about the stated hypothesis, some partial conclusions can be made. The calibration method worked, but performed poorly and can thus not be recommended at this point. The implementation of the system in ROS based on the environment interaction model worked well. \\


\section{Learned lessons}
Probably the most important lessons learned in this project is humility to the level of complexity in the proposed system and to the process of developing a complete system as first time experience with practical robotics. So many things have been a first encounter from simple online/offline considerations to working with real noisy data. In many cases idealised small-scale experiences comprised the origin for decisions about a very real full-scale robotics system making an example of the vast differences between two worlds.
