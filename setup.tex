% PAGE DIMENSIONS
\usepackage{geometry}
\geometry{a4paper,margin=3.5cm}

% PACKAGES
\usepackage[english]{babel}
\usepackage[T1]{fontenc}
\usepackage[utf8]{inputenc}
\usepackage{graphicx}
\usepackage{fancyhdr}
\usepackage{color}
\usepackage{eso-pic} %background pictures
\IfFileExists{inconsolata.sty}{\usepackage{inconsolata}}{}
\usepackage{csquotes}
\usepackage{booktabs}
\usepackage{tabularx}
\usepackage{array}
\usepackage{verbatim}
\usepackage{titlesec}
\usepackage{hyperref}
\usepackage{float}
\usepackage{caption}
\usepackage{subcaption}
\usepackage{amsmath,amsfonts,amssymb,amsthm}
\usepackage{xfrac}
\usepackage[intoc]{nomencl} %for abbreviations list
\usepackage{listings}
\usepackage{marginnote}
\usepackage{url}
\usepackage[normalem]{ulem}
\usepackage{tocbibind}
\usepackage{fancyhdr}
\usepackage{algorithm}
\usepackage{algpseudocode}

% Bibliography
\usepackage[style=numeric,natbib=true,sortcites=true,block=space,backend=bibtex]{biblatex}
\bibliography{ref/leonref}
% Use 'biber %' to build bibliography

% Table of contents
\setcounter{tocdepth}{1}

% Headers and footers
\pagestyle{fancy}
\renewcommand{\headrulewidth}{0pt}
\lhead{\slshape \leftmark}
\chead{}
\rhead{}
\lfoot{}
\cfoot{}
\rfoot{\thepage}

% marginnote package options
\renewcommand*{\marginfont}{\color{red}\sffamily} %red, sans-serif
\reversemarginpar %margin notes on left side

% to justify typewriter monofont if used a lot in one place
\newcommand*\justify{%
  \fontdimen2\font=0.4em% interword space
  \fontdimen3\font=0.2em% interword stretch
  \fontdimen4\font=0.1em% interword shrink
  \fontdimen7\font=0.1em% extra space
  \hyphenchar\font=`\-% allowing hyphenation
}

% Define custom typewriter shorthands (uses \justify)
\newcommand{\justtt}[1]{\texttt{\justify{\small #1}}}
\newcommand{\smalltt}[1]{\texttt{{\small #1}}}

% listings package settings
\lstloadlanguages{Matlab,[ANSI]C}
\lstset{
  basicstyle=\ttfamily\small,
  %aboveskip=12pt,
  %belowskip=12pt,
  %frame=l,
  numbers=left,
  numberstyle=\ttfamily\tiny,
  numbersep=8pt,
  captionpos=b,
  tabsize=2,
  extendedchars=true,
  breaklines=true,
  showspaces=false,
  showtabs=false,
  keywordstyle=\color{blue},
  escapeinside={(*¤}{*¤)},
  %rangeprefix=,
  %rangesuffix=,
  includerangemarker=false,
  %stringstyle=\color{white}\ttfamily,
  %commentstyle=\color{white} %''cheat'' to hide comments
  xleftmargin=7pt,
  xrightmargin=7pt,
  %backgroundcolor=\color{lightgray},
  showstringspaces=false,
  morekeywords={}
}

% hyperref package settings
\hypersetup{
  unicode=true,          % non-Latin characters in Acrobat’s bookmarks
  pdftoolbar=true,        % show Acrobat’s toolbar?
  pdfmenubar=true,        % show Acrobat’s menu?
  pdffitwindow=false,     % window fit to page when opened
  pdfstartview={FitH},    % fit page to the window Horizontal/Vertical
  pdftitle={report},    % title
  pdfauthor={Kent Stark Olsen},% author
  pdfsubject={An analysis and implementation},   % subject of the document
  pdfkeywords= {University of Southern Denmark} {SDU}, % list of keywords
  pdfnewwindow=true,      % links in new window
  colorlinks=false,       % false: boxed links; true: colored links
  linkcolor=red,          % color of internal links
  citecolor=green,        % color of links to bibliography
  filecolor=magenta,      % color of file links
  urlcolor=cyan,           % color of external links
  plainpages=false
}

% Widow/orphan penalties
\widowpenalty=300
\clubpenalty=300

% \includegraphics default folder
\graphicspath{{graphics/}}

% Float positioning control
\setcounter{topnumber}{2}
\setcounter{bottomnumber}{2}
\setcounter{totalnumber}{3}
\renewcommand{\topfraction}{0.85}
\renewcommand{\bottomfraction}{0.85}
\renewcommand{\textfraction}{0.15}
\renewcommand{\floatpagefraction}{0.4}

% Header colour
\definecolor{FrontpageHeadingColor}{RGB}{5,5,60}%Heading colour definition

% Chapter name formatting
\titleformat
  {\chapter}%command
  [display]%shape
  {\normalfont\huge\bfseries}%format
  {\normalfont\Large\scshape\chaptertitlename\ \huge\thechapter}%label
  {10pt}%sep
  {\Huge}%before

% make nomenclature and change its heading/ text (see nomencl package)
\makenomenclature
\renewcommand{\nomname}{Abbreviations}
%
%  * Pass this line to MakeIndex:
%    %bm.nlo -s nomencl.ist -o %bm.nls
%
%  * Use \nomenclature{abbr}{discriptive text} to add an entry to the
%    abbreviations list (best done where the abbreviation first occurs in the text)
%  * Example:
%    \nomenclature{ADHD}{Attention Deficit Hyperactivity Disorder}
